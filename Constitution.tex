\documentclass{article}
\usepackage{hyperref}
\title{Tripartite Constitution}
\date{May 2019}
\author{Daniel J. Michael Write}
\begin{document}
\maketitle
\newpage
\tableofcontents
\newpage
\section{Framework of Government}
\subsection{Deliberative Assembly}
\subsubsection{Formation}
\paragraph{Members}
The deliberative assembly shall consist of ninety-three local representatives and thirty-one party representatives.
\paragraph{Elections}
Local representatives shall be elected every fourth year in their respective districts. They shall be divided into classes such that one-fourth is elected every second year.

Party representatives shall be elected every second year in a general election. Each political party or caucus shall be awarded a number of seats proportionate to their respective shares of the total vote, but only one person shall exercise the votes of the seats of each political party at any given time.
\paragraph{Qualifications}
Each member of the deliberative assembly shall have been a citizen of the state for two years prior to the date of their elections. No person shall be elected without having attained eighteen years of age.
\paragraph{Majority and Quorum}
Except as otherwise provided by law, a two-thirds affirmative majority shall be necessary for the deliberative assembly to decide any question. A quorum shall consist of a requisite number of members for that majority. Absent members may be compelled to attend by penalties established by the rules of the deliberative assembly.
\paragraph{State Governor}
The state governor shall preside over the regular sessions of the deliberative assembly and see to its day-to-day business. The state governor shall be chosen by the local representatives in a secret election.
\subsubsection{State Executive}
\paragraph{Appointment}
The state executive shall be nominated by the state governor and confirmed by the deliberative assembly.
\paragraph{Vacancy and Succession}
Upon the vacancy of the office or at other times during which the state executive is unable to perform the duties of office, the longest-serving member of the deliberative assembly shall assume the duties of office and undertake its powers until a successor is nominated and confirmed or the incumbent state executive regains the ability to execute the powers of office. During this time, the replacement state executive shall not exercise any powers or duties within the deliberative assembly.
\paragraph{Relations with State Councils}
The state executive shall serve as an \textit{ex officio} member of each state council. In this capacity, the state executive shall exercise no vote unless the members stand equally divided. The state executive shall convene and adjourn state councils as provided by law.
\paragraph{Powers}
The state executive may call forth the militia to execute the laws of the state, to suppress insurrection, and to act in the event of a natural disaster within the state. The state executive may grant pardons and reprieves for the crimes of the state with the advice and consent of the judicial council. The state executive may convene, preside over, and adjourn special sessions of the deliberative assembly and of the state councils. The state executive shall enforce the laws of the state as the state executive councils provide.
\subsubsection{Legislative Process}
\paragraph{Legislative Committees}
For each pair of legislative and executive councils, there shall be a corresponding committee of the deliberative assembly.

The members of each committee shall be drawn from the local representatives and assigned by the state governor. Each caucus of the deliberative assembly shall be assigned an equal number of seats on each committee. 

Upon receiving a bill from its corresponding legislative council, each committee shall discuss and debate the bill. Amendments to the bill may be proposed and adopted by the committee during this time if approved by the supervisory judge of the corresponding legislative committee as complying with the purpose of the bill; decisions of the supervisory judge may be appealed to the constitutional court. The bill shall pass from the committee after ten days of receiving it unless the committee agrees to postpone, but cannot be postponed more than thirty days. The legislative process of the bill is terminated if the committee unanimously rejects the bill.
\paragraph{Open Debate}
Upon the passage of a bill from committee, the state governor shall schedule the bill for open discussion and debate for a time not less than seventy-two hours and in no less than three separate meetings of the deliberative assembly. Once the allotted discussion time has expired, the bill shall lay dormant for two days during which time representatives shall publicly issue any concerns and stand-asides or blocks. After those two days, the state governor shall schedule a vote. If the deliberative assembly fails to approve the bill, it shall vote to reject the bill. If this vote fails, a snap election shall be held.
\paragraph{Stand-asides and Blocks}
Stand-asides shall not be considered in vote tallying, quorum, or majority requirements when voting on a bill. A block causes the immediate return of a bill to its originating committee for further amendment and later re-introduction, but if blocked again, the legislative process of that bill shall terminate. A block must be sustained by more than one-third of the local representatives and may be issued only by a party representative. Each party representative shall be entitled to one non-transferable and non-renewable block per each term, but may use that block twice on the same bill.
\subsubsection{Specific Powers}
\paragraph{Subpoenas}
The deliberative assembly or any committee may compel public officers, except judges, to provide sworn testimony.
\paragraph{Question Time}
The state governor shall hold question time once a week, during which the state governor and other officers of the deliberative assembly shall answer any questions of the members of the assembly.
\paragraph{Votes of No Confidence}
The deliberative assembly may dismiss the state governor upon a successful vote of no confidence. A motion of no confidence shall be brought only by a public petition bearing the signatures of one-tenth of the members of the deliberative assembly. Votes of no confidence shall not be scheduled within six months before an election of the deliberative assembly or less than four months after a previous unsuccessful vote of no confidence.
\paragraph{Emergency Laws}
The unanimous decision of the deliberative assembly may institute laws outside of the prescribed legislative process. Laws passed in this manner shall expire after no more than six months and shall not be re-instituted as emergency law within two years after their expiration dates.
\paragraph{Impeachment}
The deliberative assembly may impeach any public officer of the state, including members of the assembly and juridical nominees as juridical nominees. Upon the adoption of an article of impeachment giving reasonable notice to the acts or omissions alleged to constitute impeachable offenses, a special state council shall be convened consisting of seven juridical nominees selected by lot. This council shall try the case against the impeached officer. The deliberative assembly shall elect two members to act as their representatives. The trial shall be conducted in the manner of civil proceedings and the impeached officer shall not be permitted to invoke protections against self-incrimination except as applicable in a general civil case. Conviction shall require the unanimous concurrence of the members of the council. Judgment shall not extend beyond removal from office and disqualification to hold any office of honor, trust, or profit of the state. The impeached officer, whether convicted or acquitted, shall be liable to prosecution and punishment pursuant to the law. The impeached officer shall not exercise the powers of office upon impeachment until acquittal.
\subsubsection{Snap Elections}
\paragraph{When Held}
Snap elections of the deliberative assembly shall be held if a budget is not approved no less than two weeks prior to the expiration of the current budget, if the deliberative assembly is unable to elect a state governor or state executive for a period of time exceeding thirty days, or as otherwise provided by law.
\paragraph{Execution of Necessary Emergency Law}
During times when the deliberative assembly is disbanded, the state councils may execute necessary and proper laws subject to final veto by the state executive. The deliberative assembly may reverse such laws when it reconvenes.
\newpage
\subsection{Courts of Law}
\subsubsection{Supreme Court of Appeals}
\paragraph{Formation}
The supreme court shall consist of forty-five judges serving life terms. Vacancies shall be filled by the state executive. These judges shall not be eligible to hold any other public office of the state after their terms.
\paragraph{Appeals from Other Courts}
When a decision from a lower court is appealed, the supreme court shall select a panel of three of its judges by lot to try the case. If such a panel finds a judgment which the judges of the panel believe to conflict with the judgment of a previous panel, the judges shall move the case for final review by the supreme court.
\paragraph{Final Review}
One-third of the judges must consent to invoke final review on a case. Cases subject to final review shall be tried by the supreme court \textit{en banc}. Cases may be appealed for final review from panel judgment if a majority of the judges of the supreme court consent. Panel judgments may be reversed or modified only if the supreme court decides by two-thirds majority or if the same question is decided by final review twice giving the same judgment. When the judges of a panel move a case for final review, the supreme court may modify or reverse the panel judgment by simple majority. The supreme court may not exercise discretion as to trying cases from panel splits.
\paragraph{Judicial Review}
The supreme court may not exercise judicial review, but in cases where a majority of judges deem that any laws in question are in violation of this constitution, the supreme court may stay proceedings and move the case to the constitutional court for judgment.
\paragraph{Jurisdiction}
The supreme court shall have jurisdiction over all appeals from the lower courts and over cases which are not in the jurisdiction of the constitutional court. The supreme court shall have general superintendent power over all courts of the state and shall prescribe rules governing the practice of law within the state and the procedures of the courts of the state. The supreme court shall govern admissions to the practice of law within the state.
\subsubsection{Constitutional Court}
\paragraph{Formation}
The constitutional court shall be formed of thirty-five judges appointed by the judicial council for a term of fifteen years. They shall not be eligible to hold any other public office of the state after this term.
\paragraph{Jurisdiction}
The constitutional court shall have original and exclusive jurisdiction on matters of disputes concerning the extent of powers vested in the instruments of the state by this constitution, of disputes concerning the compatibility of other law with this constitution, of disputes concerning the rights and duties of the state and sub-national government, of disputes concerning the proceedings of the deliberative assembly and of other procedural bodies of the state, of disputes concerning the alleged infringement of rights of a person by state authority, of disputes between two or more sub-national governments, and as otherwise assigned by law.
\paragraph{Judicial Review}
The constitutional court may rule that laws violate this constitution on their intrinsic merits and declare such laws null and void. Suits alleging as such may be brought by any sub-national government or one-fifth of the deliberative assembly. 

The constitutional court may rule that laws as they are applied violate this constitution and declare those applications null and void. Suits alleging as such must be brought only in cases of actual controversy.
\paragraph{Invocation}
The constitutional court may be invoked by any citizen, sub-national government, or group of one-fifth of the deliberative assembly providing substantive claims relating to the jurisdiction of the constitutional court.
\paragraph{Power of Dismissal}
If a judge of the state intentionally infringes upon the principles of this constitution, the constitutional court may order that judge to be immediately dismissed by a unanimous concurrence of its judges.
\paragraph{Dismissal of Judges}
Members of the constitutional court shall be privileged from impeachment, but may be dismissed by the judicial council or the court of auditors for improper character or criminal activity respectively.
\subsubsection{Other Courts}
\paragraph{Extraordinary Courts}
No extraordinary courts shall be established. Courts for particular fields of law shall be established only as provided by law.
\paragraph{Sub-national Courts}
Each sub-national government may create courts with jurisdiction over all justiciable matters within the borders of those courts. In all cases, judges shall be juridical nominees and shall not be elected.
\paragraph{Court of Auditors}
The constitutional court and the supreme court shall appoint a court of auditors to be comprised of juridical nominees not serving in other offices. The court of auditors shall have the power to investigate abuses of power and obstructions of justice by public officers and to charge those officers with those crimes. Members of the court of auditors shall be privileged from subpoena and impeachment.
\subsubsection{Judicial Norms}
\paragraph{Reporting of Cases}
All courts shall be courts of record. The decisions of all courts, along with the reasons therefor, shall be transmitted to the judicial council for official publication.
\paragraph{Judicial Qualifications}
Judges shall be juridical nominees and citizens of the state. Judges shall serve until they have attained seventy-five years of age whereupon their seat shall be declared vacant.
\paragraph{Juridical Nominees}
A juridical nominee is any person of at least twenty-five years of age who is in the possession of an accredited and well-regarded degree in law or legal studies. The judicial council shall consult with respected third-party authorities in the legal realm to suggest persons to the deliberative assembly as well-suited candidates. The deliberative assembly shall either confirm or reject each candidate, confirmed candidates shall be juridical nominees.
\newpage
\subsection{State Councils}
\subsubsection{Formation}
\paragraph{In General}
Each state council shall be established by the deliberative assembly with specific and particular competencies and powers. Unless otherwise provided, each council shall be comprised of one qualified person appointed by the state governor and one qualified person appointed by the state governor and each party representative. 
\paragraph{Qualifications}
Members of a state council shall possess expertise in the competencies of their respective councils. Expertise may be measured in degrees or experience but shall not be subject to review by courts.
\paragraph{Prohibition on Dual Office}
No person shall concurrently serve on two or more state councils.
\paragraph{Replacement of Members}
Members of a state council may be replaced by their respective appointors at their discretion.
\subsubsection{Supervisory Judge}
\paragraph{Appointment}
Each state council shall be appointed a supervisory judge by the judicial council. Supervisory judges shall exercise no vote, but may assist in the drafting of policy.
\paragraph{Veto Power}
Supervisory judges may veto adopted policy on the basis of its intrinsic merits being in violation of this constitution. These vetoes may be appealed to the constitutional court by any member of the council.
\subsubsection{Constitutional Councils}
\paragraph{Legislative and Executive Councils}
The deliberative assembly may establish pairs of legislative and executive councils to write and enforce laws respectively. The deliberative assembly may request that certain councils function co-dependently towards certain objectives or request that certain policies be adopted, but the councils shall be under no obligation to honor those requests.
\paragraph{Electoral Council}
The electoral council shall be formed in the same manner as other state councils with the addition of an equal number of qualified persons appointed by the judicial council to the sum of all other members. The electoral council shall manage all elections of the state, maintain voter rolls, register valid citizens to vote, certify and declare the results of elections, print ballots, certify candidates, and undertake other necessary duties to maintain free and fair elections in the state. The electoral council shall establish the electoral maps of the state.
\paragraph{Judicial Council}
The judicial council shall be formed in the same manner as other state councils with the addition of an equal number of persons appointed by the supreme court to the sum of all other members. All members of the judicial council shall be juridical nominees. The supervisory judge of the judicial council shall be selected by the constitutional court.
\newpage
\section{Framework of Rights}
All people are born in equal dignity, freedom, and independence. They have been endowed with reason and conscience. The rights of all people derive from their natural rights of the life, liberty, and independence of the person. 
No distinction under the law may be made between the people, including race, color, creed, religion, expression or opinion including those of a political nature, national origin or language, property, sex or gender, sexual orientation, or other statuses or conditions.
Each right is to be held sacred and any law in their violation shall be null and void. The rights and freedoms not enumerated in the proceeding are of equal importance and effect, and thus shall not be denied.
No person shall be denied recognition as an equal being or otherwise be removed from their rights. No person shall be denied the equal protection or due process of the law.
No person shall be denied the ability to seek monetary compensation in the courts for acts undertaken by government in the violation of their rights. No person shall be denied the ability to bring a lawsuit against the state except as provided by law.
No power of the suspension of laws or of this constitution shall be exercised.
No person shall be held in slavery or status of servitude. No person shall be denied the ability to seek an employment of choice.
No person shall be inflicted by obscene punishments, including torture and capital punishment, or cruel and unusual punishments. No person shall be subject to inhumane treatment.
No person shall be held for excessive bail, but bail may not be set if there is evidence or great presumption that they pose great harm to themselves or society.
No person shall be denied personal security or be subject to unreasonable interference in their home, papers, correspondence, and effects except with a warrant granted by a judge on the basis of probable cause.
No person shall be subject to the search or seizure of property without the presentation of a warrant to that person particularly describing the things to be searched or seized.
No person shall be denied the ability to keep arms for personal defense and security.
In a criminal prosecution, the accused shall enjoy a swift and public trial by an independent jury of peers, the knowledge of the accusations made and the persons making those accusations, assumed innocence, provided counsel if unable to procure legal assistance, the ability to compel favorable witnesses, the privilege against self-incrimination, and at least one appeal of a conviction.
No person shall be imprisoned on account of debt except in the case of fraud.
The writ of habeas corpus shall not be suspended. No conviction shall work corruption of blood or forfeiture of estate.
No person shall be denied the ability to own private property as an individual or as a collective, but private property may be appropriated for public works only with just compensation.
No person shall be denied the ability to freely seek and impart information, thoughts, and ideas through any media or be denied the ability to otherwise express opinion or personal statements. 
No person shall be compelled to manifest beliefs against their conscience or be restricted from manifesting their conscience. 
No system of thought or religion shall be inhibited or advanced. No religious test shall be applied against public officers.
No person shall be denied the ability to peacefully assemble, associate, and protest without prior notice or to petition the government for a redress of grievances.
No elector shall be denied the ability to vote. No payment of any form shall be required for an elector to be eligible to vote or for the registration of a person qualified to become an elector an such.
\newpage
\section{Framework of Law}
\end{document}